\section{Experiment}\label{sec:eval}
\subsection{Methodology}\label{sectn:method}
%compiler and libs
We implement our library in ISO
C++. Theoretically, any standard-compliant C++ compiler
should process our classes without trouble. New C++ standard (a.k.a C++0x\cite{c++0x}) adds many language features to ease metaprogramming\footnote{When we
  conducted this work, C++0x was close to finish. Iimplementing C++0x
  were in progress for many compilers}. Compilers
without C++0x support need some workarounds to pass compilation
though, they do not hurt expressiveness.We developed the library and
tested using GCC 4.5 beta.  The
first implementation of OpenCL was shipped by Mac OSX 10.6. The GPU performance is collected on that platform.

% Consider the trend of C++,
%development of template library like libvina should become easier
%and smoother in the future.  

A couple of algorithms are evaluated for our template approach.  They
are typical in image processing and scientific fields. In
addition, we implement a pseudo language translation program to
illustrate pipeline processing. The programs in experiments are listed
as follows:

\begin{itemize}
\item[\textit{saxpy}] Procedure in BLAS level 1. A scalar multiplies to a single precision vector, which contains 32 million elements.
\item[\textit{sgemm}] Procedure in BLAS level 3. Two 4096*4096 dense matrices multiply.
\item[\textit{dotprod}] Two vectors perform dot production. Each vector
  comprises 32 million elements.
\item[\textit{conv2d}] 2-Dimensional convolution operation on image.  The Image
  is 4094*4096 black-white format. Pixel is normalized as a single
  float ranging from 0.0 to 1.0.
\item[\textit{langpipe}] Pseudo-Multi-language translation. A word is translated from one language A to language B, and then another function will translate it from language B to language C, etc.
\end{itemize}

Two multicore platforms are used to conduct experiments. The hardware
platforms are summed up in Table.~\ref{tbl:mach}

\begin{table}[hbt]
\caption{Experimental platforms}\label{tbl:mach}
\begin{center}
\begin{tabular}{|l|l|l|l|l|r|}
\hline
\textbf{name}&\textbf{type}&\textbf{processors}&\textbf{memory}&\textbf{OS}\\
\hline
harpertown&SMP &x86 quad-core  &4G&Linux Fedora\\
                  &  server &   
2-way  2.0Ghz & &kernel 2.6.30\\
\hline
macbookpro&laptop &x86 dual-core &2G&Mac OSX\\
                    &           & 2.63Ghz         &  &Snowleopard\\
                   &           &GPU 9400m    &256M & \\
                    &           & 1.1Ghz   & &\\
\hline
\end{tabular} 
\end{center}
\end{table}
On harpertown, we link Intel Math kernels to perform BLAS procedures
except for conv2d. On macbookpro, we implemented all the algorithms on
our own. For CPU platform, we link libSPMD thread library to
perform computation. The library binds CPUs for each SPMD
thread and switch to realtime scheduler on Linux.  This configuration
helps eliminate the impact of OS scheduler and other processes in the
system.

\subsection{Evaluation}
\subsubsection{Speedup of Hierarchical transformation on CPU}
\begin{figure}
\includegraphics{speedupx86.0}
\caption{Speedup on Harpertown}\label{fig:spdx86}
\end{figure}

Fig.~\ref{fig:spdx86} shows the speedup on harpertown. The blade
server contains two quad-core Xeon
processors. We experiment hierarchical transformation for
algorithms. All predicates are set to cater to CPU's last level cache(LLC).

We obverse good performance scalability for programs
\textit{conv2d} and \textit{sgemm}. \textit{conv2d} does not have any dependences
and it can obtain about 7.3 times speedup in our experiments. \textit{sgemm}
needs an extra reduction for each division operation. The final
speedup is about 6.3 times when all the cores are available. Note that we
observe almost two-fold speedup from sequence to dual core case. But the speedup degrates to 3.3 times when executionu
environment changes to 4-core. Harpertown consists of 2-way quad-core processors,  Linux
can not guarantee that 4 subtasks are distributed in a physical
processor. Therefore, the cost of memory accesses and synchronization
increases from 2-core to 4-core.

\textit{dotprod} and \textit{saxpy} reveal low speedup because non-computation-intensive
programs are subject to memory bandwidth.  In average, \textit{saxpy} needs one load and one 
store for every two operations. \textit{dotprod} has similar
situation. They quickly saturate memory bandwidth for SMP system and
therefore perform badly. Even though we fully parallelize those
algorithms by our template library. 

\subsubsection{Speedup of  SPMD transformation on GPU}
\begin{figure}
\includegraphics{speedupgpu.0}
\caption{Speedup Comparing GPU with CPU}\label{fig:spdgpu}
\end{figure}

Fig.~\ref{fig:spdgpu} shows SPMD transformation results for GPU on
macbookpro. GPU's memory model has significantly different from
GPU. Because TF\_hierarchy makes little sense for GPU, we directly use building
block par to translate iterations into OpenCL's NDRangeKernel
function. Programs running on host CPU  in sequence are set as
baseline. Embedded GPU on motherboard contains 2
SMs\footnote{Streaming Multiprocessor, each SM consists of 8 scalar processors(SP)}.
Porting from CPU to GPU, developers only need a couple of lines to change
templates while keeping algorithms same \footnote{
Because GPU code needs special qualifiers, we did modify kernel
functions a little manually.  Algorithms are kept except for sgemm. It is not easy
 to work out sgemm for a laptop, so we added blocking and SIMD
 instruments for CPU.}. As figure depicted,  computation-intensive programs
\textit{sgemm} and \textit{conv2d} still maintain their speedups. 4.5 to 5 times
performance boost is achieved for them by migrating to GPU.
In addition, we observe about 2 times performance boost for
\textit{saxpy}. Nvidia GPUs execute
threads in group of warp (32 threads) on hardware and it is
possible to coalesce memory accesses if warps satisfy
specific access patterns. Memory coalescence mitigates bandwidth issue
occurred on CPU counterpart. Because our program of \textit{dotprod} has fixed
step to access memory which does not fit any patterns, we can not
obtain hardware optimization without tweaking the algorithm.

\subsubsection{Comparison between different multicores}
\begin{table}[hbt]
\caption{Comparison of sgemm on CPU and GPU}\label{tbl:sgemm}
\begin{tabular}{|l|r|r|r|}
\hline
& baseline& CPU & GPU\\
\hline
\textbf{Cores} &1 x86(penryn)& 8 x86(harpertown)& 2 SMs\\
\hline
\textbf{Gflops}& 2.64 &95.6&  12.0\\
\hline
\textbf{Effectiveness}&12.6\%& 74.9\%&68.2\%\\
\hline
\textbf{Lines of function}&63&unknown&21\\
\hline
\end{tabular}
\end{table}

Table.~\ref{tbl:sgemm} details \textit{sgemm} execution on CPU and GPU. Dense matrix
multiplication is one of  typical programs which have
intensive computation. Problems with this characteristic are the most
attractive candidates to apply our template-based approach.
Our template library transforms the \textit{sgemm} for both CPU and 
GPU. We choose sequential execution on macbookpro's CPU as
baseline. After mapping the algorithm to GPU, we directly obtains over
4.5 times speedup comparing with host CPU. Theoretically,  Intel Core
2 processor can issue 2 SSE instructions per cycle,  therefore, the
peak float performance is 21 Gflops on host CPU. We obtain 2.64 Gflops which
effectiveness is only 12.6\% even we employ quite complicated
implementation. On the other side, 12 Gflops is observed on GPU whose
maximal performance is roughly 17.6 Gflops.\footnote{$17.6Gflops = 1.1Ghz * 2(SM) *
  8(SP)$. nVidia declared their GPUs can perform a mad(multiply-add
  op) per cycle  for users who concern performance over precision. However, we can
  not observe mad hints bring any performance improvement in OpenCL. }
Although both column 2 and column 4 implement SIMD algorithm for
\textit{sgemm}, GPU's version is obviously easier and effective. It is
due to the dynamic SIMD and thread management from GPU
hardware~\cite{Fatahalian08} can significantly ease vector programming. Programmer can
implement algorithm in plain C and then replies on template
transformation for GPU.  Adapting to GPU only need tens of lines code
efforts. Like GPU template, we apply building blocks directly to parallelize \textit{sgemm} procedure for CPU. We observe 95.6 
Gflops and about 75\% effectiveness on harpertown server.

\subsubsection{Pipeline Transformation for CPU}
\begin{figure}[htp]
\includegraphics{pipeline.0}
\caption{Pipeline Processing for Psuedo Language Translation}\label{fig:pipe}
\end{figure}

Fig.~\ref{fig:pipe} demonstrates pipeline processing using our
template library. As described before, \textit{langpipe} simulates a
multilingual scenario. We apply template TF\_pipeline listed in
List. 2. In our case, the program consists of 4 stages,
which can transitively translate English to Chinese\footnote{follow the 
  route: English  $\to$ French $\to$ Spanish $\to$ Italian $\to$
  Chinese}. Only the preceding stages complete, it can proceed with
the next stages. The executing scenario is similar to Fig.~\ref{fig:viewmt}. We use bogus loop to consume $t \  \mu s$ on CPU. For each $t$, we iterate 500
times and then calculate the average consumptive time on harpertown. For
grained-granularity cases (20$\mu s$, 50$\mu s$, 100$\mu s$), we can obtain ideal
effectiveness in pipelining when 4 cores are exposed to the system.
\textit{i.e.} our program can roughly output one instance every $t\  \mu
s$. The speedup is easy to maintain when granularity is big. 100 $\mu s$ case ends up 54 $\mu s$ for each instance for 8 cores. 50  $\mu s$ case
bumps at 5 cores and then improves slowly along core increment. 20
$\mu s$ case also holds the trend of first two cases. 5 $\mu s$ case is
particular. We can not observe ideal pipelining until all 8
cores are available.  Our Linux kernel scheduler's granularity is 80
$\mu s$ in default. We think that the very fine granular tasks contend
CPU resources in out of the order. The runtime behavior presumably
incurs extra overhead. Many cores scenario helps alleviate the
situation and render regular pipeline processing.

