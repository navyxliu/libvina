%\section{Related work}\label{sec:related}
%\begin{figure}
%  % Requires \usepackage{graphicx}
%  \centering
%  \includegraphics[width=0.8\linewidth]{../fig/motivation}
%  \caption{CFSF Motivation}\label{fig:motivation}
%  \label{fig:algorithm}
%\end{figure}
\section{Introduction}\label{sec:Intro}
%multicore problem
Modern computer architectures rely on parallelism and memory
hierarchy to improve performance. Both duplicated processors and
elaborated storage-on-chip require programmers to be aware of underlying
machines when they write programs. Even worse,
multicore technologies have brought many architectural features for
different implementations. Thus, it is challenging to develop
efficient applications which can take advantage of various multicores.

%need new programming model
In essence, it is because plain C/C++ programming languages
can not reflect contemporary architectures~\cite{cml}. Traditionally, programmers
describe algorithms in sequential logics, and then resort to  compiler
and hardware optimization to deliver modest performance relative to
their machines. In multicore era, this classic programming model gains
little. It is desirable to develop alternatives to utilize horsepower
of multicores while hiding architectural features.

% to aggresssive is not feasible
Although researches on revolutionary programming models have obtained
fruitful achievements, they are limited in specific
domains~\cite{gmapreduce, erlang, haskell}.
One critical issue hinders them from applying in general programming
field is that one programming model can only
benefit a small group of users. It is still unclear what
general purpose programming model is. Besides, the cost hardware usually weights
a small part in a computer system relative to software and
personnel. The ratio lowers with time. Therefore, vendors are
reluctant to adopt fundamental changes of software stacks for
multicore evolvement.

%Second, the cost of hardware in large systems
%is relative low than software and personnels. The ratio lowers along time
%pass. It is risky to drop up existing efforts and build software from
%scratch using new programming models.

An acceptable tradeoff is to extend traditional programming
languages to utilize effective parallel patterns. Appearently, the advantage of
this approach is that it can exploit multicores progressively. Thus the
knowledges and
experiences of traditional programmers are still useful, and investment
of legacy software is saved. In industry, OpenMP~\cite{openmp}  and
TBB~\cite{tbb} are successful cases. OpenMP provides parallel
programming API in the form of compiler directives. TBB is a C++ library, consisting of concurrent containers and
iterators. CUDA~\cite{cuda} extends C programming languate to describe
groups of threads for GPU. The limitation of preceded approaches are
platform or vendor dependent. In academia,
Sequoia~\cite{sequoia} attempts to programming for memory hierarchy. It
achieves parallelization by divide a task into subtasks hierarchically and then map
subtasks on nodes of machines. Merge~\cite{merge} implements
map/reduce programming model for heterogeneous
multicores. Streamit~\cite{ThiesKA02} compiler supports stream/kernel
model for streaming computation. Its run-time schedules kernels for specific architectures. The shortcoming of academical approaches is that each one
is capable of one type of parallel patterns.  In a word, existing solutions
lack uniform method to express multiple parallel patterns
across various multicores.

%FIXME: move to related works
%It(merge) relieson hierarchical division of task and predicate-based
%dispatch system to assign subtasks on matched multicore target at
%runtime. OpenMP and TBB are for shared memory system
%system such as SMP, while CUDA is property programming
%model for specific GPU architectures.

Observably, except TBB is a pure library-based solution, aforementioned
approaches need compilers to facilite their programming models. It is
the ad-hoc approaches embedded into compilers restrict flexibility and
extensibility. Therefore, we propose a template-based programming model to
support parallel programs for multicores. We exploit C++
metaprogramming techniques to perform source-to-source transformation
in the unit of functions. We use \emph{tasks} to abstract
computation-intensive and side-effect free functions, which are candidates for transformations. We extend the meaning of template
specialization~\cite{tcpl}. Our approach specializes a task for target's
architectures.  Through applying template classes, a task is
transformed into many subtasks according to different parallel
patterns, and then subtasks are executed in the form of threads.
Template classes are implemented for different multicore
architectures. As a result, porting software from one platform to another only
needs to adjust template parameters or change implementations of
template classes. The difference between TBB and our approach is that
we utilize C++ template metaprogramming, so the transformations complete
at compile time. 

%pros and cons
Our approach is flexible and extensible. Both parallel patterns and
execution models are provided as template classes, thus programmers can
parallelize tasks using more than one way. In addition, template
classes are organized as template library. It is
possible to exploit architectural
features and new parallelization strategies by extending library. We
explore language features limited in ISO
standard C++~\cite{c++98, c++03, c++0x}, so it is applicable for
platforms with standard-compilant compilers. Most 
platform-independent template classes can be reused. The limitation of
template-based approach is that using template metaprogramming, only compile-time
information is available. That includes static constant values,
constant expression and type information in C++. Therefore, our
approach is not a general solution and orients for programs with rich
static information. Fortunately, it is not uncommon that this
restriction is satisfied in the fields like embedded applications and
scientific computation. Because the runtime of those programs with fixed
parameters are significantly longer than compile time even the time to
write programs, it will pay off if can resolve transformations at
compile time. Besides, it is possible to utilize external
tuning framework~\cite{tuningfrm} to adjust parameters of static programs.

In summary, we proposed a template-based programming model, which 
tailors programs to multicores. Programers apply template classes to
transform functions into the parallel equivalences on source-level,
and then map them on specific multicores to run simultaneously.

%organization structure
The remaining parts of this paper are structured as follows.
Section~\ref{sec:model} presents our programming model. Section~\ref{sec:lib} introduces libvina
-- a prototype library to facilitate template-based programming model.
Section~\ref{sec:adaption} is how programers adapt their source codes to libvina. Section~\ref{sec:details} gives details of
implemetation of our library. Section~\ref{sec:eval}
evaluates performance on both CPU and GPU using our approach.
Section~\ref{sec:related} summarizes related work to support parallel programs
for multicores. Section~\ref{sec:con} is disscussion and future work.

