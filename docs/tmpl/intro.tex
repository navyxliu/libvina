%\section{Related work}\label{sec:related}
%\begin{figure}
%  % Requires \usepackage{graphicx}
%  \centering
%  \includegraphics[width=0.8\linewidth]{../fig/motivation}
%  \caption{CFSF Motivation}\label{fig:motivation}
%  \label{fig:algorithm}
%\end{figure}


\section{Introduction}\label{sec:Intro}
%multicore problem
Modern computer architectures rely on parallelism and memory
hierarchy to improve performance. Both duplicated processors and
elaborated storage-on-chip require programmers to aware of underlying
machines' knowledge when they write programs. More worse,
multicore technologies have brough various architectural features for
different implementations. Therefore, it is challenge to develop
efficient applications which can take advantage of various multicores.
%need new programming model
In essence, we think it is because plain C/C++ programming language
can not reflect comtemporary architecture. Traditionally, programmers
describe algorithms in seqential logics, and then reply on compiler
and hardware optimization to deliver modest performance in terms of
their machines. In muticore era, this classic prorgramming model gain
little. It is desireable to develop alternatives to utilize powerhorse
of multicores while hiding architectural features.
% to aggresssive is not feasible
Although researches on revolutionary programming models have otained
fruitful achievements, there still are critical issues to prohibit
changing. First, it is still unsolved problem what a general purpose programming
languages is. Considering the wide specrum of parallel computers, we
doubt it really exists. Second, the cost of hardware in large systems
is relative low than software and personnels. The ratio lowers along time
pass. It is risky to drop up existing efforts and build software from
scratch using new programming models.

An acceptable tradeoff is to extend traditional programming
languages to utilize proved achievements. The obvious advantage of
this approach is that it can exploit multicores progressively. The knowldges and
experiences of tranditional programmers are still useful and legancy
software are saved. In industry, OpenMP~\cite{openmp}  and
TBB~\cite{tbb} are successful cases.OpenMP provides parallel
programming API in the form of compiler directives for shared memory
systems. TBB consists of containers and iterations.

Source-to-source transformation can help tailor specific
architectures. In particular, some transformations can extend
traditional programming language to support multicore
architectures. OpenMP~\cite{openmp} and Sequoia~\cite{sequoia, sequoia-compiler}
are typical examples. One distinct advantage comparing with other
fancy languages is that they
support progressive parallelization from original source code, which
can guarantee to keep software investment. OpenMP transform program
regions into fork-join threads based on pragma directives. Sequoia attempts to map computation-intensive functions
on a machine tree describing in configuration file. In those
extended languages,  a set of language
constructs are provided to support specific parallel patterns, so they
need dedicated compilers. However, the ad-hoc approaches of
source-to-souce transformation can not support general parallel
patterns. It is not a trivial task to determine how many language
constructs should be provided by compilers to well support the
full spectrum of multicores.

%what we do and how
We present an approach to perform source-to-souce transformation to
support multicore using C++ template mechanism. We uses \emph{tasks}
to abstract computation-intensive and side-effect free function. Our
primary idea is to extend C++ template specialization to task for
different multicores.  Template
classes can transform a task into other forms according the parallel
patterns and then map to threads to execute in parallel. 

The primary limitation is that only static
information are available at compile time. Therefore, it only works
for programs which own rich static information. Fortunately,
applications with this characteristic are pervasive in multimedia,
digital processing, and scientific computation.  Because template takes effect at compile time, it is possible to avoid
deploying run-time for transformation, which means that it can incur minimal
runtime cost. Besides, our template-based approach imposes fewer restricts comparing with other static approaches:

\begin{itemize}
\item Flexibility: We proposed a way to perform source-to-source transformation by
metaprogramming. Because it can manipulate source code in metaprograms,  our approach does not bind any parallel models. It is easy to
change transform to fork-join, or perform computation
as pipeline. In addition, our approach can deploy any thread
implementations to support parallelism and concurrency. We experiment
pthread, low-level threads provided by OS and device driver. As far as we know, no parallel programming language
declares such flexibility. Theoretically, metaprogramming is
as expressive as any general-purposed programming languages, so we
think it is a promising approach to explore more parallel patterns
beyond this paper.

\item Extendability:  It is extensible to develop new template class to
  utilize new architectural features  and parallel patterns. Template
  metaprogramming is intimate for C++ programmers. It is easy to
  extend new execution models and parallel patterns. Other approaches
  have to ratify languages and then modify compiler to complete
  features. The progress is usually a year-old compaign and can not
  determined by software developers alone.

\item Portability: Template is part of ISO C++~\cite{c++03, c++0x}. Template-based approach is applicable for every
  platforms with standard C++ compiler. Template metaprogramming is
  widely used in other applications in C++ community and full-blown
  metaprogramming libraries like
  MPL~\cite{mpl} is portable. Through good encapsulation of
  platform details, most of code in our template approach can be reused.
\end{itemize}

%organization structure
The remaining parts of this paper are structured as follows.
 Section 2 shows techniques to perform
transforms by template metaprogramming. Audiences with C++ template
programming experiences or functional programming language concepts
are helpful but not prerequisites. Then Section 3 presents some
typical transformations by our template library. Experiments are in
Section 4 to evaluate performance on both CPU and GPU.
Section 5 summarizes some related works on library-based approach and
language extension to support multicore architectures.  The last section is conclusion and future work.

