\section{Related work}\label{sec:related}
%As mentioned before, it is desirable to extend conventional
%programming languages to reflects new hardwares. Researches in the field have two major directions:
Programming models to support parallel programs for muticores can be
broadly categorized into directions:
\begin{enumerate}
\item providing library to support programming for concurrency
\item extending language constructs to extend parallel semantics
\end{enumerate}

Library is a common method to extend language capability
without modifying grammars. Pthread library is a \textit{de facto} standard for
multi-threading on POSIX-compatible systems. The relationship between
pthread and native thread is straightforward. Therefore, abstractions
of pthread are far away from expressing parallelism naturally. The same problem occurs on OpenCL or other vender-dependent
libraries for GPUs. Libvina is a metaprogramming library instead of
system library. We provides high-level parallel patterns
and executions as template classes. Implementations take
responsibility for binding tasks on specific muticore platforms. C++
community intends to develop parallel libraries while bearing generic programming in mind. TBB  has a plenty of
containers and building blocks to support loop-parallism and
task-level parallelism.  Inspired by TBB's approach, we enable the
same effects in static domain. We aim at utilizing static information
to perform source
transformations for different architectures at compile time. Besides, template-based approach we propose is orthogonal to runtime parallel
library TBB. We only explore parallelism which can be resolved by compilers, developers feel free to deploy TBB to
farther improve programs.


%Furthermore, the implementation of thread on hardware is
%undefined in the standard, so it can not guarantee performance or even
%correctness on some architectures \cite{Boehm05}.

%Entities including partitioner and
%scheduler in TBB are created at run time. In that case, key data
%structures have to be thread-safe. Although TBB exploits task
%parallelism or other sophisticated concurrency on general purpose
%processors, the runtime overhead is relative high in data parallel
%programs, especially in the scenario that many lightweight threads are
%executing by hardware. 


%% MPI
%Another dominant parallel library is MPI in supercomputing community. It based on message-passing mechanism and SPMD model to execute parallel program. The difficulties of developing MPI programs are as notorious as pthread counterpart for programmers without sufficent training of parallel programming. 
%%

%OpenMP ~\cite{openmp} is designed for shared memory and has been shipped in almost every C/Fortran
%compilers.
%OpenMP can only perform Fork-Join parallel model.
%Source-to-Source transformation for optimization was reported by~\cite{Loveman77}, whose granularity of transformation  is
%statement.  Most of works have been merge in forms of IR in
%modern compilers. New source-to-source transformation compilers focus
%on function. 
%The run-time is usually provides in the form of dynamic link library.
%Although it is simple and
%portable, the performance is not optimal in most cases. Moreover, a
%handful of directives in OpenMP leave little room for further improving
%performance or scaling up to larger systems. Hybrid OpenMP with MPI is
%possible though, difficulties surge. 

The second direction for language community is to extend language constructs by
modifying compiler. They add language constructs for compiler to
express parallellism. OpenMP~\cite{openmp} compilers transform sequential code blocks into
multi-threaded equivalences based on
directives. OpenMP is \textit{de facto} standard for shared memory
though, the programming model does not fit heterogeneous mulitcores. Sequoia~\cite{sequoia, sequoia-compiler} supports
programming memory hierarchy. In order to achieve portability for
parallel programs, a source-to-source compiler transforms a \textit{task} into a cluster of
\emph{variants}, and then maps variants on tree-style virtual 
machines, which are described by external configuration files.
We derive the same idea to choose implementations at compile time for
different architectures. Merge~\cite{merge} is a map/reduce programming
framework for heterogeneous multicore 
systems in the forms of task and variant. It relies on hierarchical division of task and predicate-based
dispatch system to assign subtasks on matched multicore target at
runtime. Each approach mentioned above can complete a kind of parallel pattern.
We demonstrate our template-based approach can achieve the same
functionalities using template metaprogramming if parameters are
available at compile time. 

We intend to fuse the advantages of pure library approach and
specialized parallel programming languages. Extending 
languages to express parallelism usually needs to modify compilers. We
think it is this process hardwires fixed parallel patterns into the
compilers. Therefore, we explore the powerness of metaprogramming to
transform sources for parallelism, which
can support mulitple parallel programming models while maintain
portability for mulitcores.

%First of all, it targets execution
%environment as a tree of machines, which an individual machine owns
%its storage and computation unit. Second, Target machine
%is described in XML files. \cite{sequoia} reports that Sequoia can transform programs
%for CellBE, cluster while keeping competitive performance. That
%is at expense of implementing one compiler for each platforms.
%The primary drawback of Sequoia is that its language constructs can not cover common
%parallel patterns such as pipeline or task queue. Methods mentioned before all need non-trivial efforts to
%modify compilers. As discussed in \cite{sequoia}, the authors of the Sequoia were
%still not clear whether the minimal set of primitives they provided provides can
%sufficiently express dynamic applications. We doubt if it is worthwhile to
%invest a compiler given the fact that template library can also
%achieve the same functionalities.

