\section{Related work}\label{sec:related}
In this section, we briefly review previous work, including
memory-based, model-based and other CF approaches.

\subsection{Memory-based CF approaches}
Memory-based CF approaches can be further classified into item-based,
user-based and UI-based approaches according to the rating source used
in prediction~\cite{Xuegr:CF@2005, Sarwar:WWW01@2001}. Item-based CF
approaches predict user preferences from the ratings made on similar
items by the same user~\cite{Linden:Amazon@2003}. User-based CF
approaches exploit the idea that like-minded users may go for the same
item. Alternatively, UI-based CF approaches predict user preferences
not only from the ratings used by item-based and user-based CF
approaches, but also from the ratings that the like-minded users make
on similar items.

Representative memory-based CF approaches are SF~\cite{wangjun:CF@2006}
and EMDP~\cite{Ma2007@SIGIR}. SF is one type of UI-based CF approach
that unifies the item-based and user-based approaches. SF considerably
improves prediction accuracy, but it is slow due to a search on similar
items and like-minded users over the whole item-user matrix. SF
predicts user preferences without capturing the diversity in user
ratings. This diversity is incorporated in EMDP, which calculates the
items and users whose similarities exceed certain thresholds. Thus,
EMDP improves the accuracy and returns exact results. EMDP is based on
a set of different thresholds for each item and user. This is a
computer-intensive job, due to the large quantity of items and users in
recommender systems. Moreover, inappropriate thresholds may lead to few
results, leaving users expectant to receive suggestions from the
feedback system. EMDP tends to select the most similar items and
like-minded users for prediction, which is easily achieved using CFSF.

\subsection{Model-based CF approaches}
Model-based CF approaches learn a model of user ratings in an item-user
matrix, and then use it to predict the scores of unrated items for an
active user. They use a variety of techniques to create the model, such
as Bayesian network~\cite{Zhang:Bayes@2007}, linear
classifiers~\cite{Tong@JMLR2002} and Principal Component Analysis
(PCA)~\cite{Linden:Amazon@2003}. Typical examples for model-based CF
approaches are clustering~\cite{Connor:SIGIR@01} and aspect
model~\cite{cheung:extendedLSM@2004}.

Usually, model-based CF approaches are fast to predict user preferences
because they narrow down the search space for identifying similar items
and like-minded users. They tend, however, to suffer from
time-consuming training and updating processes. Furthermore, most of
them do not consider the ratings made on similar items by like-minded
users~\cite{Xuegr:CF@2005, HOFMANN@TOIS2004}.

\subsection{Other CF work}
Several other types of CF approaches have also been
proposed~\cite{Koren@2008KDD, Klein@JCSS, jin2006smm, Robert@2007ICDM,
Pennock:Personality@2000}. In~\cite{Rennie@ICML05, Robert@KDD07},
matrix factorization is explored to improve prediction accuracy of CF.
In~\cite{Xuegr:CF@2005}, SCBPCC is a cluster-based CF approach that
introduces a smoothing strategy among user clusters and efficiently
solves the data sparsity problem. It achieves high levels of accuracy,
but doesn't capture the ratings of like-minded users on similar items.
Moreover, SCBPCC could be further improved in scalability because it
identifies the similar items over the entire item-user matrix each
time.

In contrast, content-based CF approaches, used in Google and Yahoo!,
predict user preferences based on the classifications of item contents,
rather than item sources or other criteria~\cite{Gediminas:TKDE@2005,
Raymond:BookRecom@2000}. They are, however, limited by two assumptions:
1) users are able to express their particular preferences or
information requirements with respect to intrinsic features of items,
and 2) systems are capable of thoroughly comprehending item contents
and accurately extracting their features~\cite{HOFMANN@TOIS2004}.
