\section{Implementation}
\label{sec:details}

We implement all the functionalities described before using C++
template metaprogramming technique.
%The grand idea is to utilize
%template specialization and recursion to achieve control flow at
%compile time. Besides template mechanism, other  C++ high level
%abstracts act important roles in our approach.
Function objects and bind mechnism are commonly employed to postpone computation
at proper place with proper enviroment~\cite{moderncpp}.
%In order to utilize nested buiding blocks,
%lambda expression can generate closure objects in a concise
%form, e.g., line 24$\sim$37 of \reffig{sgemm}.

Implemenation of building blocks are straightforward. We use recursive calls
to support nesting patterns. \code{seq} and \code{par} are interoperable
because we chose properly nested classes before calling function \code{apply}. Note that
building blocks do not know the nesting levels during the execution. To solve this
problem, each function object or cloure object is decorated with a loop-variable
counter.
%The handlers take responsibility for calculating loop variables in
%normalized form. It is only desirable for nested loop forms,
%\textit{e.g.} line 41 of List 1.
Because some callable objects in C++ such as clousure object
do not provide default constructors, we pass their
references in those cases. 
%Consequently, some call sites of building blocks are different from \reftable{bb}.

For CPU, building blocks are implemented by embedding OpenMP directives. On GPU, we bind
building block classes to functions of OpenCL~\cite{opencl}, an open standard API
for heterogenous muliticores. The \code{mt::thread} is implemented using pthread.
For the signal mechanism among views, conditional variables of pthread are used
for CPU and events are employed for GPU.

%end section
