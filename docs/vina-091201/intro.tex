\section{Introduction}
\label{sec:Intro}

%multicore problem
Modern computer architectures rely on parallelism and memory
hierarchy to improve performance. Both duplicated processors and
elaborated storage-on-chip require programmers to be aware of underlying
machines when they write programs. Nowadays,
multicore technologies have brought many architectural features for
different implementations. Thus, it is challenging to develop
efficient parallel applications that can take advantage of various multicores.

%need new programming model
In essence, it is because plain C/C++ programming languages
cannot reflect contemporary architectures~\cite{cml}. Traditionally, programmers
describe their algorithms using sequential logics, and then resort to compiler
and hardware optimization for best performance that their machines can achieve.
In the multicore era, this model apparently does not take full advantage of the
hardware. It is imperative to develop alternatives to exploit horsepower
of multicores while hiding architectural features.

% to aggressive is not feasible
Although researches on revolutionary programming models have obtained
fruitful achievements, they are suitable in specific
domains~\cite{gmapreduce, erlang, haskell}.
One critical issue hinders them from being applied in general programming
field is that one programming model can only
benefit a small group of users.
%Besides, the cost of hardware usually occupies a small ratio of the whole computer system relative
%to the cost of software and personnel. The ratio lowers with time. Therefore,
System vendors are reluctant to adopt fundamental changes of software stacks for
multicore evolvement.

%It is still unclear what general purpose programming model is.
%Second, the cost of hardware in large systems
%is relative low than software and personnel. The ratio lowers along time
%pass. It is risky to drop up existing efforts and build software from
%scratch using new programming models.

An acceptable tradeoff is to extend traditional programming
languages to facilitate effective parallel patterns. The advantage of
this strategy is that it can exploit multicores progressively based on
legacy code sources. Thus the knowledge and
experiences of traditional programmers are still useful, and investment
of legacy software is saved. OpenMP~\cite{openmp}  and
CUDA~\cite{cuda} are successful cases. OpenMP provides parallel
programming API for C/Fortran in the form of compiler directives. CUDA
extends the C programming language to describe
groups of threads for GPU. The limitation is that those approaches
depend on specific platforms or vendors.

On the other side, many experimental
programming models are proposed to support multiple architectures in recent years.
Sequoia~\cite{sequoia} attempts to program for memory hierarchy. It
achieves parallelization by hierarchically dividing a task into subtasks
and then mapping subtasks on physical machines. Merge~\cite{merge} implements
MapReduce programming model for heterogeneous
multicores. Streamit~\cite{ThiesKA02} compiler supports stream/kernel
model for streaming computation. These attempts often target at one type of
parallel patterns. Thus, their expressiveness is limited.
%shortcoming of the experimental attempts is that each one
%is only capable of one type of parallel patterns.  In a word, existing solutions
%lack a uniform method to express multiple parallel patterns
%across various multicores.

%Its run-time schedules kernels for specific architectures.
%TBB is a C++ library, consisting of concurrent containers and
%iterators. 
%FIXME: move to related works
%It(merge) relieson hierarchical division of task and predicate-based
%dispatch system to assign subtasks on matched multicore target at
%runtime. OpenMP and TBB are for shared memory system
%system such as SMP, while CUDA is property programming
%model for specific GPU architectures.

%Observably, except TBB is a pure library-based solution, aforementioned
%approaches need compilers to facilitate their programming models. It is
%the ad-hoc approaches embedded into compilers restrict flexibility and
%extensibility. Therefore, 

We propose a template-based programming model to
support parallel programs for multicores. We exploit the C++
metaprogramming techniques~\cite{tempmetaprog} to perform source-to-source transformations
in the unit of functions. We use \emph{tasks} to abstract
computation-intensive and side-effect free functions.
By extending C++ template
specialization~\cite{tcpl}, our approach specializes a task for targeted
architectures. Our template classes can transform a task into many subtasks
according to different parallel patterns, and then subtasks are executed in
parallel in the form of threads.

%The difference between TBB and our approach is that
%we utilize C++ template metaprogramming, so the transformations complete
%at compile time. 

%pros and cons
We have implemented our approach in a template library called libvina. For 
many applications in embedded and scientific fields, there are rich
static information, such as static constant values, constant
expression, and type information in C++ programs.
We have transformed many typical procedures in these fields into their parallel
equivalences at source-level using libvina. Our evaluation shows that the
transformed code can run different multicores with significant speedups, while
porting software from one platform to another only
needs to adjust template parameters or change implementations of
template classes at compile time. 

%It is possible to utilize external
%tuning framework~\cite{tuningfrm} to adjust parameters of static programs.

\comment{
Our approach is flexible and extensible. Both parallel patterns and
execution models are provided as template classes, thus programmers can
parallelize programs using more than one way, targeting to multiple architectures. In addition, template
classes are organized as a library. It is
easy to exploit architectural
features or apply appropriate parallelism strategies for specific applications
by extending the library.

We explore language features limited in the ISO
standard C++~\cite{c++98, c++03, c++0x}, therefore our approach is
applicable for every platforms with standard-compliant compilers. Most 
platform-independent template classes can be reused.
} 

%The runtime of those programs
%with fixed parameters are significantly longer than the time of
%compilation,  even
%the time of writing the programs, so it will pay off if we can resolve
%parallelization at compile time.

The rest of this paper is organized as follows.
Section~\ref{sec:model} presents our programming model. 
Section~\ref{sec:adaption} uses two examples to present how to use libvina.
Section~\ref{sec:details} details the implementation of our library.
Section~\ref{sec:eval} evaluates performance on both CPU and GPU using our approach.
Section~\ref{sec:related} summarizes related work to support parallel programs
for multicores.
Finally, Section~\ref{sec:con} concludes with future work.

