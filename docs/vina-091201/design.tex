\section{Template-based Programming model}
\label{sec:model}

\begin{figure}[htb]
\includegraphics[width=3.3in, height=4.0in]{../overview}
\caption{Overview of template-based programming model: Programmers
write side-effect free functions in C/C++, then encapsulate them in
template classes. Such a template encapsulate the computing function and can be 
automatically transformed into a
group of subtasks by TF classes based on appropriate parallel patterns. Finally,
subtasks directly run on physical multicores.}
\label{fig:overview}
\end{figure}

We propose a template metaprogramming approach to support parallel programs
running on different multi-core systems. \reffig{overview} gives an overview
of our approach: a side-effect free function is abstracted as a \emph{task}
and is wrapped in a template class. We provide template classes
to support different parallel patterns:

\begin{itemize}
\item \textbf{Hierarchy pattern.} A task is recursively divided into subtasks
and subtasks can execute on multiple cores in parallel.

\item \textbf{Pipeline pattern.} A task is divided into a series of processing
subtasks, where the output of a previous subtask is directly used as the input
of the next subtask and each subtask execute on a core.

\item \textbf{MapReduce pattern.} A task is divided into a map phase and a
reduce phase, where subtasks in each phase can execute in parallel on multiple
cores. The input of reduce phase is the output of the map phase.

\end{itemize}

The transformation for a task from sequential code to parallel equivalence is a
source-to-source conversion with C++ templates, which is enabled by a \emph{TF
class} (Section~\ref{sect:tf}).  The converted code calls
architecture-specific building block classes (Section~\ref{sect:bb}) so that
when compiled, a task can execute on
different multicore systems in parallel at run time. 

The rest of this section first describes three class types of our approach, i.e.,
TF, view, and building block classes. 
%(1) View class, a representation of underlying containers such as vector or matrix.
%(2) Building block class,
%provide basic executions of tasks on multicores (3) TF class, each one
%represents a parallel pattern. 

\subsection{TF Classes}
\label{sect:tf}

A \textit{TF class} (short name for \textit{TransFormation class}) is a template
class representing a parallel pattern which transforms a task to a group of
subtasks in isomorphism. In other words, the transformed task has the same interface
while owns a call graph inside to complete the original computation by a
group of subtasks. Specifically, the following two classes are used in this paper: 

\begin{itemize} 
\item \textbf{TF\_hierarchy}. This template class recursively divides a task into subtasks
until certain predicate is evaluated as true. %As Fig.~\ref{fig:mmexample} depicted,  we use
In fact, \code{TF\_hierarchy} can implement a programming model similar to Sequoia and
we use \code{TF\_hierarchy} to implement both hierarchy and MapReduce pattern.

\item \textbf{TF\_pipeline}. This template class synthesizes a call chain of an
arbirary number of functions into a pipeline. This is a common pattern for
stream kernel programming model.
\end{itemize} 

\subsubsection{TF\_hierarchy}

\reffig{tf:code} illustrates the class definition of \code{TF\_hierarchy}. The
prime template (line 1 to 10) takes  user supplied \code{TASK}
class and a predicate class as parameters. Note the third parameter, the bool value,
is calculated from the predicate class at compile time, so programmers don't
use this parameter. 
The predicate class is similar to Merge~\cite{merge} and is used to generate
control flow for recursions. The main difference from Merge is that our predicate is a
\emph{metafunction} and is evaluated in place to obtain the bool value, which
depends on the two types, \code{TASK::ARG0} and \code{TASK::ARG1}
defined in the class \code{TASK}. Usually these two types are views (Section~\ref{sec:view})
with statically known size. When the third parameter is evaluated as false, the
prime template is used and \code{doit} function invokes recursive
\code{TASK:inner} function, which divides a task into subtasks.
On the other hand, when the third parameter is evaluated as true, the partial
specialization template (line 12 to 19) is used, where \code{doit} function
invokes \code{TASK::leaf} function (leaf node, no recursion).


\begin{figure}[hbt]
  \inputsrc{tf.cc}
  \caption{Pseudo-code of class \code{TF\_hierarchy}.}
  \label{fig:tf:code}
\end{figure}

\begin{figure}[hpt]
  \includegraphics[width=3.0in]{../algo}
  \caption{Instantiation process of \code{TF\_hierarchy}. The predicate is a template
class, which is evaluated using \code{TASK}'s parameters.}
  \label{fig:hierarchy}
\end{figure}

The instantiation process of \code{TF\_hierarchy} is illustrated in \reffig{hierarchy}:
when predicate is false, the task is recursively divided into subtasks; when predicate
is true, the code (\code{TASK::leaf}) computes a subtask.

%A side-effect free function is referred to as \emph{task} in
%libvina. As a rule of thumb,
%nFIXME
\comment{In this paper, we consider a class of computation-intensive functions (or tasks) that are 
self-contained, i.e., external data references are limited and
calling graphs of them are simple. For these functions, it's possible to
decouple a task into a cluster of subtasks. These subtasks are identical
except for arguments and we can distribute subtasks on multi-core to execute simultaneously. 
}

%We implement two TF classes in libvina though  it is not
%necessary to use TF classes to perform source transformations. We
%encourage to do so because it has engineering advantages, which reduces
%effects of system programmers.

\subsubsection{TF\_pipeline}
The \code{TF\_pipeline} class using variadic template~\cite{vartemp}, as
illustrated in \reffig{pipeline:code}. \code{TF\_pipeline} supports an arbitrary
number of functions chaining together and the only limitation is the maximal
level of template recursions of a compiler.

%For C++ compilers don't support variadic template, there
%are workarounds to achieve the same effect, but quite
%tedious.\footnote{zhangsq ask me to cite. I implemented the
 % workarounds myself, but i don't see it is necessary to show them here. too
 % details... --xliu 28. Nov}.

\begin{figure}[hbt]
  \inputsrc{pipeline.cc}
  \caption{Pseudo-code of class \code{TF\_pipeline}.}
  \label{fig:pipeline:code}
\end{figure}


\subsection{View Classes}
\label{sec:view}

A \emph{View} is a class representing a subset of container's data. For example,
a matrix type could have views that contains a subset of elements of a matrix.
There are two kinds of views, \code{ReadView} and \code{WriteView}.
A \code{ReadView} is read-only, while a \code{WriteView} allows write operations on its data
(by providing interfaces to write). 

To ensure multi-thread safety, \code{ReadViewMT} and \code{WriteViewMT} are
defined. For these two types, each object contains a signal that is copied across multiple
threads. All operations of \code{ReadViewMT} or \code{WriteViewMT} are blocked until the
current thread is signaled by other threads.

\reffig{view} depicts the relationship of views. Concrete lines means that
a type cast from a source node to a destination node is legal, i.e., an implicit
conversion in C++.  Dashed lines means a source node can generate objects of
the type of the destination node. Labels on the dashed line signifies how
signals are created or copied.
Shadow region is another thread space.

%The only approach to communicate with other
%threads is through a special kind of view called \emph{ViewMT}.  

The design of view class serves two purposes. 1) The classes are type-safe.
Because template instantiation is invisible for programmers, our
source transformations by templates may introduce subtle errors. We
expect compilers complain explicitly when unintentional
transformations happen. For example, when a \code{ReadView} is accidently used
for writes, the compiler will complain for errors. 2) View classes hide architecture specifics. %communication details. 
Implementations have choices to optimize data movement according to
architectures. Shared memory systems~\cite{larrabee} and communication-exposed
multicores~\cite{cellbe, imagine} usually have different strategies to perform
the operations. In fact, our implementation of for CPU and GPU are different (See
Section~\ref{sec:details}).

\begin{figure}
\includegraphics[width=3.6in, height=2.5in]{../relationship_views}
\caption{The relationship of view classes. A concrete line represents a valid
   type cast, while a dashed line represents a source node can generate objects
of the type pointed to. Labels specify how signals are created or copied
across views.}
\label{fig:view}
\end{figure}


\subsection{Building Block Classes}
\label{sect:bb}

\begin{table}[hbt]
\caption{Building block classes in libvina}
\begin{tabular}{|c|l|l|}
\hline
Name& Semantics& Usage Example\\
\hline
\textbf{par$<$T, K, F$>$}& Iterate function \textit{F} \textit{K} times &par$<$\_tail, 4, F$>$\\ 
                         &in parallel, implicit barrier                 &::apply();\\                
\hline
\textbf{seq$<$T, K, F$>$}& Iterate function \textit{F} \textit{K} times&seq$<$\_tail, 5, F$>$\\
                         &                                             &::apply();\\
\hline
\textbf{reduce$<$K, F$>$}&Reduce \textit{K} values using &reduce$<$8, F$>$\\
&function
\textit{F}&::apply(values)\\
\hline
\textbf{mt::thread$<$F$>$}&Spawn a thread to execute  & mt::thread$<$F$>$\\
                          &function \textit{F}        & ::apply();\\
%\textbf{do\_}&loop until predicate is evaluated true.& 
\hline
\end{tabular}\label{tbl:bb}
\end{table}

A building block class is a high-level abstraction of executions that
hide architecture-specific details. Programmers can use building blocks to
execute tasks on different multi-core platforms.
\reftable{bb} lists building blocks in libvina. For tasks that can
executed in SPMD (Single-Program-Multiple-Data) fasion, \code{par} class spawns
multiple threads to execute iterations in parallel. \code{seq} class 
execute iterations in order. Parameter $T$ in \code{par} or \code{seq} could be nested
\code{par}, nested \code{seq}, or \code{\_tail} (meaning no further nestings).
For instance, we can write the following statement 

%However, if it is
%not the case, we have to deal with dependences carefully using \code{seq} and \code{reduce}. 
%Like traditional programming languages, our
%building blocks of iterations support nesting definition. In addition, both \textit{seq} and
%\textit{par} are interoperable. 

\begin{lstlisting}
seq<par<_tail, 4>, 3, F>::apply();
\end{lstlisting}
to build to a two-level loop, and the nested loop are executed in
parallel. The equivalent OpenMP code is:
\begin{lstlisting}
F f;
int i, j;
for (i=0; i<3; ++i)
{
  #pragma omp parallel private(j)
  for (j=0; j<4; ++j) 
    f(i, j);
}//implicit barrier
\end{lstlisting}

%The first template parameter $T$ of iterations is used to support nest. It could be
%either a par or a seq. Special classes \code{par\_tail} and \code{seq\_tail} are
%symbols to indicate the end of nest.

The \code{reduce} class supports the ``reduce'' pattern in MapReduce style tasks.
Specifically, a given function $F$ is used to reduce $K$ input values and the
final result is stored in the first value. Note that many of the $K-1$ times
reduce operations are executed in parallel using multiple threads.

Finally, \code{mt::thread} class provides a thread interface for programmers to
dynamically spawn a new thread to execute a function.

%can exploit it to bind thread directly (\textit{e.g.} line 19 of List 2)
%or develop other customized building blocks.



%programming model
\comment{
We use template metaprogramming to implement a parallel programming model.
Essentially, our approach utilizes C++ template mechanism to
perform source-to-source transformations for multicores. Side-effect
free functions are abstracted as \emph{tasks}. A task is
wrapped in the form of template class, named \emph{function
  wrapper}.  A \emph{TF class} is a template class, which
is capable of transforming a task into a group of subtasks based on
a parallel pattern. Tasks
apply TF classes according to their appropriate
parallel patterns. This process is called ast \emph{adaption}. 
Finally, we use \emph{building block classes} to define executions of tasks
on specific architectures. Both TF classes and building
blocks are organized as a library --
\textbf{libvina}.}


